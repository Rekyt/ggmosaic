%%%%%%%%%%%%%%%%%%%%%%%%%%%%%%%%%%%%%%%%%
% Large Colored Title Article
% LaTeX Template
% Version 1.1 (25/11/12)
%
% This template has been downloaded from:
% http://www.LaTeXTemplates.com
%
% Original author:
% Frits Wenneker (http://www.howtotex.com)
%
% License:
% CC BY-NC-SA 3.0 (http://creativecommons.org/licenses/by-nc-sa/3.0/)
%
%%%%%%%%%%%%%%%%%%%%%%%%%%%%%%%%%%%%%%%%%

%----------------------------------------------------------------------------------------
%	PACKAGES AND OTHER DOCUMENT CONFIGURATIONS
%----------------------------------------------------------------------------------------

\documentclass[DIV=calc, paper=a4, fontsize=10pt, twocolumn]{scrartcl}	 % A4 paper and 11pt font size

\usepackage{lipsum} % Used for inserting dummy 'Lorem ipsum' text into the template
\usepackage{url}
\usepackage{natbib}
\usepackage[english]{babel} % English language/hyphenation
\usepackage[protrusion=true,expansion=true]{microtype} % Better typography
\usepackage{amsmath,amsfonts,amsthm} % Math packages
\usepackage[svgnames]{xcolor} % Enabling colors by their 'svgnames'
\usepackage[hang, small,labelfont=bf,up,textfont=it,up]{caption} % Custom captions under/above floats in tables or figures
\usepackage{booktabs} % Horizontal rules in tables
\usepackage{fix-cm}	 % Custom font sizes - used for the initial letter in the document
\usepackage{natbib}
\usepackage{sectsty} % Enables custom section titles
\allsectionsfont{\usefont{OT1}{phv}{b}{n} \fontsize{12}{12}} % Change the font of all section commands
\usepackage{hyperref}
\usepackage{float}
\usepackage{fancyhdr} % Needed to define custom headers/footers
\pagestyle{fancy} % Enables the custom headers/footers
\usepackage{lastpage} % Used to determine the number of pages in the document (for "Page X of Total")

% Headers - all currently empty
\lhead{}
\chead{\emph{Flying Etiquette}}
\rhead{}

% Footers
\lfoot{}
\cfoot{}
\rfoot{\footnotesize Page \thepage\ of \pageref{LastPage}} % "Page 1 of 2"

\renewcommand{\headrulewidth}{0.0pt} % No header rule
\renewcommand{\footrulewidth}{0.4pt} % Thin footer rule

\usepackage{lettrine} % Package to accentuate the first letter of the text
\newcommand{\initial}[1]{ % Defines the command and style for the first letter
\lettrine[lines=3,lhang=0.3,nindent=0em]{
\color{black}
{\textsf{#1}}}{}}
\usepackage{color}
\definecolor{purple}{rgb}{.4,0,.8}
\newcommand{\hh}[1]{{\color{magenta} #1}}
\newcommand{\st}[1]{{\color{purple} #1}}

%----------------------------------------------------------------------------------------
%	TITLE SECTION
%----------------------------------------------------------------------------------------

\usepackage{titling} % Allows custom title configuration

\newcommand{\HorRule}{\color{black} \rule{\linewidth}{1pt}} % Defines the gold horizontal rule around the title

\pretitle{\vspace{-50pt} \begin{flushleft} \HorRule \fontsize{15}{15} \usefont{OT1}{phv}{b}{n} \color{black} \selectfont} % Horizontal rule before the title

\title{Using the \texttt{ggmosaic} Package: Get That Baby Out of Here} % Your article title
 % This should probably reference geomnet. Don't want it to be too long though
\posttitle{\par\end{flushleft}\vskip 0em} % Whitespace under the title

\preauthor{\begin{flushleft}\large \vspace{-.5cm} \usefont{OT1}{phv}{b}{sl} \color{black}} % Author font configuration

\author{Haley Jeppson, } % Your name

\postauthor{\footnotesize \usefont{OT1}{phv}{m}{sl} \color{Black} % Configuration for the institution name
Iowa State University % Your institution

\par\end{flushleft} \vspace{-.5cm} \HorRule \vspace{-1cm}} % Horizontal rule after the title
\date{} % Add a date here if you would like one to appear underneath the title block

%----------------------------------------------------------------------------------------

\usepackage{Sweave}
\begin{document}
\Sconcordance{concordance:flying.tex:flying.Rnw:%
1 90 1 1 0 45 1}


\maketitle % Print the title

\thispagestyle{fancy} % Enabling the custom headers/footers for the first page

%----------------------------------------------------------------------------------------
%	ABSTRACT
%----------------------------------------------------------------------------------------

% The first character should be within \initial{}
\vspace{-1cm}
%\initial{U}\textbf{sing a new way to visualize network data in \texttt{R} with \texttt{gglot2}, I examine the evolution of the African slave trade from the $16^{th}$ through the $19^{th}$ centuries. \st{XXX I feel like I need more here. Will add more when paper is more fleshed out.}}

%\initial{T}\textbf{he Trans-Atlantic Slave Trade Database, hosted by Emory University, contains information on nearly 35,000 voyages of slave ships from 1514 - 1866 between Europe, Africa, and the Americas. The entire database contains 279 variables with information on the 34,948 voyages. While the website dedicated to this data contains a dashboard to subset and visualize the data, I wanted to view the data in a new way. I use the \texttt{geomnet} package because I wanted to show that the user can visually explore the data in a way that leaves them with a deeper understanding of the structure of the slave trade. I start by visualizing all of the data I pulled from the database on one map, then I look at different subsets of it, and I end with a much deeper understanding of the slave trade and its impact on the world.}
%----------------------------------------------------------------------------------------
%	ARTICLE CONTENTS
%----------------------------------------------------------------------------------------
\section*{Introduction}

\par The  Database, hosted by , contains
%------------------------------------------------

\section*{The \texttt{ggmosaic} package}

\par To view the   Database in a new way, I use the \texttt{ggmosaic} pac

\section*{Visualizing the data}

I start by visualizing all of the data I pulled from the database on one map. This

\section*{Conclusion}

After exploring
%----------------------------------------------------------------------------------------
%	REFERENCE LIST
%----------------------------------------------------------------------------------------
%\bibliographystyle{abbrv}
%\bibliography{ASA_SCSG_Paper}
\section*{Further Reading}
To view more visualizations of the

%----------------------------------------------------------------------------------------

\end{document}
